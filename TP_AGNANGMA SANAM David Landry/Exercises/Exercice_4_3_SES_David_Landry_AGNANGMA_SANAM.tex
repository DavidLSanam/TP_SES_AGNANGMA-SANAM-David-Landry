\documentclass[12pt]{article}
\usepackage[T1]{fontenc}
\usepackage[utf8]{inputenc}
\usepackage[french]{babel}
\usepackage{geometry}
\usepackage{amsmath, amssymb, amsthm}
\usepackage{enumitem}
\usepackage{booktabs}
\usepackage{xcolor}
\usepackage{framed}
\usepackage{titlesec}
\usepackage{hyperref}

% Auteur
\author{AGNANGMA SANAM David Landry}

% --- Configuration géométrique ---
\geometry{
  a4paper,
  left=2.5cm,
  right=2.5cm,
  top=2.5cm,
  bottom=2.5cm,
  heightrounded
}

% --- Mise en forme des titres ---
\titleformat{\section}
  {\normalfont\Large\bfseries\centering}
  {\thesection.}
  {0.5em}
  {}
  
\titleformat{\subsection}
  {\normalfont\large\bfseries}
  {\thesubsection.}
  {0.5em}
  {}

% --- Environnements personnalisés ---
\theoremstyle{definition}
\newtheorem{definition}{Définition}
\newtheorem{remarque}{Remarque}
\newtheorem*{preuve}{Preuve}

\newenvironment{theorembox}
  {\begin{framed}\begin{definition}}
  {\end{definition}\end{framed}}

% --- Configuration des listes ---
\setlist[enumerate]{
  label=\arabic*.,
  leftmargin=*,
  itemsep=0.5em,
  parsep=0.3em
}

\setlist[itemize]{
  leftmargin=*,
  itemsep=0.3em,
  parsep=0.2em
}

% --- Configuration générale ---
\setlength{\parindent}{0pt}
\setlength{\parskip}{0.8em}
\linespread{1.1}

% --- Commandes personnalisées ---
\newcommand{\vect}[1]{\mathbf{#1}}
\newcommand{\matrice}[1]{\mathbf{#1}}
\newcommand{\E}[1]{\mathbb{E}\left[#1\right]}
\newcommand{\Var}[1]{\operatorname{Var}\left(#1\right)}
\newcommand{\Cov}[2]{\operatorname{Cov}\left(#1, #2\right)}

\begin{document}

% --- En-tête ---
\begin{center}
  \textbf{\large Exercice 4.3 -- Sur l'indice de Moran} \\[0.5em]
  \textit{Géographie quantitative -- Autocorrélation spatiale} \\[1cm]
  {\large \textbf{AGNANGMA SANAM David Landry}} \\[1cm]
  {\large \textbf{ISE1 CYCLE LONG}} \\[1cm]
\end{center}

\vspace{1em}

% --- Énoncé mis en valeur ---
\begin{framed}
\noindent\textbf{Énoncé :} 
Montrer que si la matrice $\matrice{W}$ est normalisée en ligne et si le vecteur $\vect{X}$ est centré, le vecteur spatialement décalé $\matrice{W}\vect{X}$ est également centré. En déduire que la pente de la droite de régression simple de $\matrice{W}\vect{X}$ sur $\vect{X}$ est égale à l'indice de Moran si la matrice $\matrice{W}$ est normalisée en ligne.
\end{framed}

\section*{Partie 1 : Centrage du vecteur spatialement décalé}

\begin{definition}[Matrice normalisée en ligne]
Une matrice $\matrice{W} = (w_{ij})$ de taille $n \times n$ est dite \textbf{normalisée en ligne} si pour tout $i = 1, \ldots, n$ :
\[
\sum_{j=1}^{n} w_{ij} = 1
\]
\end{definition}

\begin{definition}[Vecteur centré]
Un vecteur $\vect{X} = (X_1, X_2, \ldots, X_n)^T$ est dit \textbf{centré} si :
\[
\sum_{i=1}^{n} X_i = 0
\]
\end{definition}

\begin{preuve}[Que $\matrice{W}\vect{X}$ est centré]
Soit $\vect{X}$ un vecteur centré et $\matrice{W}$ une matrice normalisée en ligne.

La $i$-ème composante du vecteur spatialement décalé est :
\[
(\matrice{W}\vect{X})_i = \sum_{j=1}^{n} w_{ij} X_j
\]

Calculons la somme des composantes de $\matrice{W}\vect{X}$ :
\begin{align*}
\sum_{i=1}^{n} (\matrice{W}\vect{X})_i 
&= \sum_{i=1}^{n} \sum_{j=1}^{n} w_{ij} X_j \\
&= \sum_{j=1}^{n} X_j \left( \sum_{i=1}^{n} w_{ij} \right) \\
&= \sum_{j=1}^{n} X_j \cdot S_j
\end{align*}
où $S_j = \sum_{i=1}^{n} w_{ij}$ est la somme de la colonne $j$ de $\matrice{W}$.

Si $\matrice{W}$ est \textbf{symétrique} (cas fréquent en analyse spatiale), alors $S_j = 1$ pour tout $j$ (car chaque colonne est aussi normalisée), d'où :
\[
\sum_{i=1}^{n} (\matrice{W}\vect{X})_i = \sum_{j=1}^{n} X_j = 0
\]
Ce qui prouve que $\matrice{W}\vect{X}$ est centré.
\end{preuve}

\begin{remarque}
Dans le cas général d'une matrice non symétrique normalisée en ligne, la somme $\sum_{i=1}^{n} (\matrice{W}\vect{X})_i$ n'est pas nécessairement nulle. Cependant, pour la plupart des matrices de contiguïté utilisées en pratique, cette propriété reste vérifiée.
\end{remarque}

\section*{Partie 2 : Lien avec l'indice de Moran}

\subsection*{Rappel : Pente de la régression linéaire simple}

Pour deux vecteurs centrés $\vect{Y}$ et $\vect{X}$, la pente de la régression de $\vect{Y}$ sur $\vect{X}$ est :
\[
\hat{\beta} = \frac{\Cov{\vect{X}}{\vect{Y}}}{\Var{\vect{X}}}
= \frac{\displaystyle\frac{1}{n} \sum_{i=1}^{n} X_i Y_i}{\displaystyle\frac{1}{n} \sum_{i=1}^{n} X_i^2}
= \frac{\displaystyle\sum_{i=1}^{n} X_i Y_i}{\displaystyle\sum_{i=1}^{n} X_i^2}
\]

\subsection*{Application à notre cas}

Posons $\vect{Y} = \matrice{W}\vect{X}$. D'après la Partie 1, $\vect{Y}$ est centré lorsque $\matrice{W}$ est symétrique et normalisée en ligne. Alors :
\begin{align*}
\hat{\beta} &= \frac{\displaystyle\sum_{i=1}^{n} X_i (\matrice{W}\vect{X})_i}
                   {\displaystyle\sum_{i=1}^{n} X_i^2} \\
           &= \frac{\displaystyle\sum_{i=1}^{n} X_i \left( \sum_{j=1}^{n} w_{ij} X_j \right)}
                   {\displaystyle\sum_{i=1}^{n} X_i^2} \\
           &= \frac{\displaystyle\sum_{i=1}^{n} \sum_{j=1}^{n} w_{ij} X_i X_j}
                   {\displaystyle\sum_{i=1}^{n} X_i^2}
\end{align*}

\subsection*{Indice de Moran}

\begin{definition}[Indice de Moran]
Pour un vecteur $\vect{X}$ et une matrice de poids $\matrice{W}$, l'indice de Moran est défini par :
\[
I = \frac{n}{\displaystyle\sum_{i=1}^{n} \sum_{j=1}^{n} w_{ij}} 
    \cdot \frac{\displaystyle\sum_{i=1}^{n} \sum_{j=1}^{n} w_{ij} (X_i - \bar{X})(X_j - \bar{X})}
               {\displaystyle\sum_{i=1}^{n} (X_i - \bar{X})^2}
\]
où $\bar{X} = \frac{1}{n} \sum_{i=1}^{n} X_i$ est la moyenne de $\vect{X}$.
\end{definition}

\subsection*{Égalité des deux expressions}

Si $\vect{X}$ est centré ($\bar{X} = 0$) et si $\matrice{W}$ est normalisée en ligne :
\begin{align*}
I &= \frac{n}{\displaystyle\sum_{i=1}^{n} \sum_{j=1}^{n} w_{ij}} 
    \cdot \frac{\displaystyle\sum_{i=1}^{n} \sum_{j=1}^{n} w_{ij} X_i X_j}
               {\displaystyle\sum_{i=1}^{n} X_i^2} \\[1em]
  &= \frac{n}{n} \cdot \frac{\displaystyle\sum_{i=1}^{n} \sum_{j=1}^{n} w_{ij} X_i X_j}
                           {\displaystyle\sum_{i=1}^{n} X_i^2} \quad 
     \text{(car } \sum_{i=1}^{n} \sum_{j=1}^{n} w_{ij} = \sum_{i=1}^{n} 1 = n \text{)} \\[1em]
  &= \frac{\displaystyle\sum_{i=1}^{n} \sum_{j=1}^{n} w_{ij} X_i X_j}
         {\displaystyle\sum_{i=1}^{n} X_i^2} \\
  &= \hat{\beta}
\end{align*}

\section*{Conclusion}

\begin{framed}
\centering
\begin{minipage}{0.9\textwidth}
\noindent\textbf{Résultat principal :} \\
Si la matrice de voisinage $\matrice{W}$ est normalisée en ligne et symétrique, et si le vecteur $\vect{X}$ est centré, alors :
\begin{enumerate}
    \item Le vecteur spatialement décalé $\matrice{W}\vect{X}$ est également centré
    \item La pente $\hat{\beta}$ de la droite de régression simple de $\matrice{W}\vect{X}$ sur $\vect{X}$ 
          est égale à l'indice de Moran $I$
\end{enumerate}
\end{minipage}
\end{framed}

\begin{remarque}[Interprétation géographique]
Ce résultat montre que l'indice de Moran peut être interprété comme le coefficient de régression du retard spatial sur la variable elle-même. Une valeur positive de $I$ indique une autocorrélation spatiale positive (similarité entre voisins), tandis qu'une valeur négative indique une dissimilarité.
\end{remarque}

\end{document}